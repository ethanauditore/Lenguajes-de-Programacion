\documentclass[answers]{exam}

\usepackage[spanish]{babel}
\usepackage{amsmath}

\newcommand{\materia}{Lenguajes de Programación}
\newcommand{\tarea}{Puntos extra}

\title{
  \huge \materia{} \\[0.5cm]
  \LARGE \tarea{} }

\author{José Ethan Ortega González}

\renewcommand{\solutiontitle}{\noindent\textbf{Solución:}\par\noindent}
\runningheadrule{}
\runningheader{\materia{}}{\tarea{}}{\today}
\footer{}{Página \thepage\ de \numpages}{}
\pointpoints{pt}{pts}

\begin{document}
\maketitle{}
\thispagestyle{headandfoot}
\begin{questions}
  \question[1] Realizar el cálculo del producto $3 \times 3$ usando las definiciones
  del cálculo lambda.
  \begin{solution}
    \begin{align*}
      (\lambda x.\lambda y.\lambda a.x(ya))33
      &\to_{\beta} (\lambda y.\lambda a.3(ya))3 \\
      &\to_{\beta} \lambda a.3(3a) \\
      &=_{def} \lambda a.3((\lambda s.\lambda z.s(s(sz)))a) \\
      &\to_{\beta} \lambda a.3(\lambda z.a(a(az))) \\
      &=_{def} \lambda a.((\lambda s.\lambda z.s(s(sz)))(\lambda z.a(a(az)))) \\
      &\to_{\beta} \lambda a.(\lambda z.(\lambda z.a(a(az)))((\lambda z.a(a(az)))((\lambda z.a(a(az)))z))) \\
      &\to_{\beta} \lambda a.(\lambda z.(\lambda z.a(a(az)))((\lambda z.a(a(az)))(a(a(az))))) \\
      &\to_{\beta} \lambda a.(\lambda z.(\lambda z.a(a(az)))(a(a(a(a(a(az))))))) \\
      &\to_{\beta} \lambda a.\lambda z.a(a(a(a(a(a(a(a(az)))))))) \\
      &\equiv_{\alpha} \lambda s.\lambda z.s(s(s(s(s(s(s(s(sz)))))))) \\
      &=_{def} 9
    \end{align*}
  \end{solution}

  \question[1] Realizar la tabla de verdad de la conjunción usando las
  definiciones de cálculo lambda.
  \begin{gather*}
    \land =_{def} \lambda x.\lambda y.xyF
  \end{gather*}
  \vspace{-3em}
  \begin{solution}
    \begin{align*}
      &\land FF &
      &\land FT &
      &\land TF &
      &\land TT \\
      &=_{def} (\lambda x.\lambda y.xyF)FF &
      &=_{def} (\lambda x.\lambda y.xyF)FT &
      &=_{def} (\lambda x.\lambda y.xyF)TF &
      &=_{def} (\lambda x.\lambda y.xyF)TT \\
      &\to_{\beta} (\lambda y.FyF)F &
      &\to_{\beta} (\lambda y.FyF)T &
      &\to_{\beta} (\lambda y.TyF)F &
      &\to_{\beta} (\lambda y.TyF)T \\
      &\to_{\beta} FFF &
      &\to_{\beta} FTF &
      &\to_{\beta} TFF &
      &\to_{\beta} TTF \\
      &=_{def} (\lambda x.\lambda y.y)FF &
      &=_{def} (\lambda x.\lambda y.y)TF &
      &=_{def} (\lambda x.\lambda y.x)FF &
      &=_{def} (\lambda x.\lambda y.x)TF \\
      &\to_{\beta} (\lambda y.y)F &
      &\to_{\beta} (\lambda y.y)F &
      &\to_{\beta} (\lambda y.F)F &
      &\to_{\beta} (\lambda y.T)F \\
      &\to_{\beta} F &
      &\to_{\beta} F &
      &\to_{\beta} F &
      &\to_{\beta} T \\
    \end{align*}
  \end{solution}
\end{questions}
\end{document}
