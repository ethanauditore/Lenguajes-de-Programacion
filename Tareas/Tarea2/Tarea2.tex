\documentclass[answers]{exam}

\usepackage[spanish]{babel}
\usepackage{minted}

\begin{document}
\begin{questions}
  \question{Se quieren representar árboles binarios cuyos únicos nodos
    etiquetados (con elementos de cualquier conjunto) son las hojas. Para ello
    se utiliza la siguiente definición recursiva de árboles:}
\begin{minted}[frame=lines]{newlisp}
(define (any? a) #t)

(define-type Arbol
  [hoja (a any?)]
  [mkt (t1 Arbol?) (t2 Arbol?)])
\end{minted}
  \begin{parts}
    \part{Definir las funciones recursivas nh, nni que calculan el número de
      hojas y el número de nodos internos (los que no son hojas) en un árbol
      respectivamente.}
    \part{Demostrar que \verb|(nh t) = (+ (nni t) 1)|}
  \end{parts}
  \begin{solution}
    \begin{parts}
      \part Las funciones son las siguientes:
\begin{minted}[frame=lines]{newlisp}
(define (nh arbol)
  (match arbol
    [(hoja _) 1]
    [(mkt i d) (+ (nh i) (nh d))]))

(define (nni arbol)
  (match arbol
    [(hoja _) 0]
    [(mkt i d) (add1 (+ (nni i) (nni d)))]))
\end{minted}
        \part{Aquí va la demostración}
    \end{parts}
  \end{solution}

  \question{Dibuja un mapa mental que muestre las fases de generación código
    ejecutable, sus principales características y elementos involucrados.}

  \question{Dadas las siguientes expresiones de WAE en sintaxis concreta, da su
    respectiva representación en sintaxis abstracta por medio de los Árboles de
    Sintaxis Abstracta correspondientes. En caso de no poder generar el árbol,
    justificar.
    \begin{parts}
      \part{\verb|{+ 18 { - 15 {+ 40 5}}}|}
      \part{\verb|{+ { - 15 {+ 40}}}|}
      \part{
\begin{verbatim}
{with {a 2}
   {with {b {+ a a}}
      {+ a {- b 5}}}}
\end{verbatim}}
    \end{parts}}
  \begin{solution}
    \begin{parts}
      \part{\verb|(add (num 18) (sub (num 15) (add (num 40) (num 5))))|}
      \part{No se puede general el árbol, porque falta un argumento en la suma
        donde solo está el 40 y en la suma principal.}
      \part
\begin{verbatim}
(with 'a (num 2)
   (with 'b (add (id 'a) (id '))
      (add (id 'a) (sub (id 'b) (num 5)))))
\end{verbatim}
    \end{parts}
  \end{solution}

  \question{Currifica cada uno de los siguientes términos:}
    \begin{parts}
      \part{$\lambda abc.abc$}
      \part{$\lambda abc.\lambda cde.acbdce$}
      \part{$(\lambda x.(\lambda xy.y) (\lambda zw.w)) (\lambda uv.v)$}
    \end{parts}
  \begin{solution}
    \begin{parts}
      \part{$\lambda a.\lambda b.\lambda c.abc$}
      \part{$\lambda a.\lambda b.\lambda c.(\lambda c.\lambda d.\lambda e.acbdce)$}
      \part{$(\lambda x.(\lambda x.\lambda y.y) (\lambda z.\lambda w.w)) (\lambda u.\lambda v.v)$}
    \end{parts}
  \end{solution}

  \question{Para cada uno de los siguientes términos, aplica $\alpha$-conversiones para
    obtener términos donde todas las variables de ligado sean distintas.}
    \begin{parts}
      \part{$\lambda x.\lambda y.(\lambda x.y \lambda y.x)$}
      \part{$\lambda x.(x (\lambda y.(\lambda x.x y) x))$}
      \part{$\lambda a.(\lambda b.a \lambda b (\lambda a.a b))$}
    \end{parts}
    \begin{solution}
      \begin{parts}
        \part{$\lambda x.\lambda y.(\lambda x.y \lambda y.x)$}
        \part{$\lambda x.(x (\lambda y.(\lambda x.x y) x))$}
        \part{$\lambda a.(\lambda b.a \lambda b (\lambda a.a b))$}
    \end{parts}
    \end{solution}
\end{questions}
\end{document}
