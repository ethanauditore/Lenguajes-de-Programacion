\documentclass[answers]{exam}

\usepackage[spanish]{babel}
\usepackage{minted}
\usepackage{amsmath}
\usepackage{amsthm}
\usepackage{geometry}

\usepackage{tikz}
\usetikzlibrary{mindmap,backgrounds}

\geometry{top=25mm,left=15mm,right=15mm,a4paper}

\newcommand{\materia}{Lenguajes de Programación}
\newcommand{\tarea}{Tarea 2}

\title{
  \huge \materia{} \\[0.5cm]
  \LARGE \tarea{} }

\author{José Ethan Ortega González}

\renewcommand{\solutiontitle}{\noindent\textbf{Solución:}\par\noindent}
\runningheadrule{}
\runningheader{\materia{}}{\tarea{}}{\today}
\footer{}{Página \thepage\ de \numpages}{}

\begin{document}
\maketitle{}
\thispagestyle{headandfoot}
\begin{questions}
  \question{Se quieren representar árboles binarios cuyos únicos nodos
    etiquetados (con elementos de cualquier conjunto) son las hojas. Para ello
    se utiliza la siguiente definición recursiva de árboles:}
\begin{minted}[frame=lines]{newlisp}
(define (any? a) #t)

(define-type Arbol
  [hoja (a any?)]
  [mkt (t1 Arbol?) (t2 Arbol?)])
\end{minted}
  \begin{parts}
    \part{Definir las funciones recursivas nh, nni que calculan el número de
      hojas y el número de nodos internos (los que no son hojas) en un árbol
      respectivamente.}
    \part{Demostrar que \verb|(nh t) = (+ (nni t) 1)|}
  \end{parts}
  \begin{solution}
    \begin{parts}
      \part{Las funciones son las siguientes:}
\begin{minted}[frame=lines]{newlisp}
(define (nh arbol)
  (match arbol
    [(hoja _) 1]
    [(mkt i d) (+ (nh i) (nh d))]))

(define (nni arbol)
  (match arbol
    [(hoja _) 0]
    [(mkt i d) (add1 (+ (nni i) (nni d)))]))
\end{minted}
      \part
      \begin{proof}
        Lo demostraremos por inducción sobre \verb|t|
        \begin{itemize}
          \item \textbf{Base de inducción:} \verb|t| = \verb|(hoja a)|. Tenemos
                lo siguiente:

                \mint{newlisp}|(nh (hoja a)) = 1 = (+ 0 1) = (+ (nni t) 1)|
                Por lo que queda demostrada la base de inducción.
          \item \textbf{Hipótesis de inducción:} Supongamos que se cumple para
                $t_{1}$ y $t_{2}$.
          \item \textbf{Paso inductivo:} Tenemos que

                \mint{newlisp}|(nh t) = (+ (nh i) (nh d))| Pero por hipótesis de
                inducción tenemos lo que sigue:

                \mint{newlisp}|(+ (nh i) (nh d)) = (+ (+ (nni i) 1) (+ (nni d) 1))|
                Y esto, por propiedades de la suma, es igual a:

                \mint{newlisp}|(+ (+ (nni i) 1) (+ (nni d) 1)) = (+ (nni i) (nni d) 2)|

                Por lo tanto,
                \mint{newlisp}|(nh t) = (+ (nni i) (nni d) 2)|

                Ahora, tenemos que:
                \mint{newlisp}|(+ (nni t) 1) = (+ (add1 (+ (nni i) (nni d))) 1)|

                Y esto, es igual a:
                \mint{newlisp}|(+ (add1 (+ (nni i) (nni d))) 1) = (+ (nni i) (nni d) 2)|

                Por lo tanto,
                \mint{newlisp}|(+ (nni t) 1) = (+ (nni i) (nni d) 2)|

                Por ultimo, concluimos que:
                \mint{newlisp}|(nh t) = (+ (nni t) 1)|
                Por lo que queda demostrado.
        \end{itemize}
      \end{proof}
    \end{parts}
  \end{solution}

  \newpage{}

  \question{Dibuja un mapa mental que muestre las fases de generación código
    ejecutable, sus principales características y elementos involucrados.}
  \begin{solution}
    El mapa mental es el siguiente:

    \scalebox{0.9}{
      \begin{tikzpicture} [ mindmap, concept color = blue!40, every node/.style
        = {concept}, every extra concept/.append style = { concept color =
          teal!60 }, grow cyclic, level 1/.append style = { concept color =
          violet!40, level distance = 4.5cm, sibling angle = 90 }, level
        2/.append style = { concept color = purple!50, level distance = 4.5cm,
          sibling angle = 45, }, ]
        \node (root) {Generación de Código Ejecutable} child {node (lex)
          {Análisis Léxico} child {node (a1) {Se descompone una cadena en
              lexemas}} child {node (a2) {Se procesan individualmente o en
              conjunto}} child {node (a3) {Programa encargado del análisis:
              analizador léxico (parser)}} } child {node (sint) {Análisis
            Sintáctico} child { node (b1) {Revisa que el programa cumpla con las
              reglas sintácticas establecidas.}} child {node (b2) {Se construye
              una representación para la computadora} } child {node (b3) {Árbol
              de Sintaxis Abstracta (ASA)} } child {node (b4) {Programa
              encargado del análisis: analizador sintáctico (parser)} } } child
        {node (sem) {Análisis Semántico} child {node (c1) {Revisa que el
              programa tenga sentido}} child { node (c2) {Si tiene sentido,se le
              da un significado a lo escrito por el programador.}} child {node
            (c3) {Programa encargado del análisis: analizador semántico}} child
          {node (c4) {Se mandará el resultado al interprete o compilará el
              código}} } child {node (trans) {Trans\-formaciones} child {node
            (d1) {Se transforma el código escrito por el programador}} child
          {node (d2) {Tiene fin de cambiar la estructura del código.}} child
          {node (d3) {Facilita alguno de los análisis anteriores y/o
              posteriores}} child {node (d4) {Eliminar azúcar sintáctica,
              verificación de tipos, etc.}} };

        \node [extra concept] at (-4,0) {La generación de código ejecutable
          empieza en el análisis léxico};

        \path (root) to[circle connection bar switch color=from (blue!40) to
        (violet!40)] (lex); \path (root) to[circle connection bar switch
        color=from (blue!40) to (violet!40)] (sint); \path (root) to[circle
        connection bar switch color=from (blue!40) to (violet!40)] (sem); \path
        (root) to[circle connection bar switch color=from (blue!40) to
        (violet!40)] (trans);

        \path (lex) to[circle connection bar switch color=from (violet!40) to
        (purple!50)] (a1); \path (lex) to[circle connection bar switch
        color=from (violet!40) to (purple!50)] (a2); \path (lex) to[circle
        connection bar switch color=from (violet!40) to (purple!50)] (a3);

        \path (sint) to[circle connection bar switch color=from (violet!40) to
        (purple!50)] (b1); \path (sint) to[circle connection bar switch
        color=from (violet!40) to (purple!50)] (b2); \path (sint) to[circle
        connection bar switch color=from (violet!40) to (purple!50)] (b3); \path
        (sint) to[circle connection bar switch color=from (violet!40) to
        (purple!50)] (b4);

        \path (sem) to[circle connection bar switch color=from (violet!40) to
        (purple!50)] (c1); \path (sem) to[circle connection bar switch
        color=from (violet!40) to (purple!50)] (c2); \path (sem) to[circle
        connection bar switch color=from (violet!40) to (purple!50)] (c3); \path
        (sem) to[circle connection bar switch color=from (violet!40) to
        (purple!50)] (c4);

        \path (trans) to[circle connection bar switch color=from (violet!40) to
        (purple!50)] (d1); \path (trans) to[circle connection bar switch
        color=from (violet!40) to (purple!50)] (d2); \path (trans) to[circle
        connection bar switch color=from (violet!40) to (purple!50)] (d3); \path
        (trans) to[circle connection bar switch color=from (violet!40) to
        (purple!50)] (d4);

        \begin{pgfonlayer}{background}
          \draw [concept connection] (lex) edge (sint) (sint) edge (sem) (sem)
          edge (trans);
        \end{pgfonlayer}
      \end{tikzpicture}
    }

  \end{solution}

  \newpage{}

  \question{Dadas las siguientes expresiones de WAE en sintaxis concreta, da su
    respectiva representación en sintaxis abstracta por medio de los Árboles de
    Sintaxis Abstracta correspondientes. En caso de no poder generar el árbol,
    justificar.
    \begin{parts}
      \part{\verb|{+ 18 { - 15 {+ 40 5}}}|}
      \part{\verb|{+ { - 15 {+ 40}}}|}
      \part{
\begin{verbatim}
{with {a 2}
   {with {b {+ a a}}
      {+ a {- b 5}}}}
\end{verbatim}}
    \end{parts}}
    \vspace{-1em}
  \begin{solution}
    \begin{parts}
      \part{\verb|(add (num 18) (sub (num 15) (add (num 40) (num 5))))|}
      \part{No se puede general el árbol, porque falta un argumento en la suma
        donde solo está el 40 y en la suma principal.}
      \part
\begin{verbatim}
(with (id 'a) (num 2)
   (with (id 'b) (add (id 'a) (id 'a))
      (add (id 'a) (sub (id 'b) (num 5)))))
\end{verbatim}
    \end{parts}
  \end{solution}

  \question{Currifica cada uno de los siguientes términos:}
    \begin{parts}
      \part{$\lambda abc.abc$}
      \part{$\lambda abc.\lambda cde.acbdce$}
      \part{$(\lambda x.(\lambda xy.y) (\lambda zw.w)) (\lambda uv.v)$}
    \end{parts}
    \vspace{-1ex}
  \begin{solution}
    \begin{parts}
      \part{$\lambda a.\lambda b.\lambda c.abc$}
      \part{$\lambda a.\lambda b.\lambda c.(\lambda c.\lambda d.\lambda e.acbdce)$}
      \part{$(\lambda x.(\lambda x.\lambda y.y) (\lambda z.\lambda w.w)) (\lambda u.\lambda v.v)$}
    \end{parts}
  \end{solution}

  \question{Para cada uno de los siguientes términos, aplica $\alpha$-conversiones para
    obtener términos donde todas las variables de ligado sean distintas.}
    \begin{parts}
      \part{$\lambda x.\lambda y.(\lambda x.y \lambda y.x)$}
      \part{$\lambda x.(x (\lambda y.(\lambda x.x y) x))$}
      \part{$\lambda a.(\lambda b.a \lambda b (\lambda a.ab))$}
    \end{parts}
    \vspace{-1ex}
    \begin{solution}
      \begin{parts}
        \part{Tenemos lo siguiente:
          \begin{align*}
            \lambda x.\lambda y.(\lambda x.y \lambda y.x) &\equiv_{\alpha} \lambda x.\lambda y.(\lambda x.y \lambda y.x) [y:=w] \\
                                  &\equiv_{\alpha} \lambda x. \lambda w.(\lambda x.w. \lambda y.x) [x:=z] \\
                                  &\equiv_{\alpha} \lambda x. \lambda w.(\lambda z.w. \lambda y.z)
          \end{align*}}
        \part{Tenemos lo siguiente:
          \begin{align*}
            \lambda x.(x (\lambda y.(\lambda x.x y) x)) &\equiv_{\alpha} \lambda x.(x (\lambda y.(\lambda x.x y) x)) [x:=w] \\
                                      &\equiv_{\alpha}\lambda x.(x (\lambda y.(\lambda w.w y) x))
          \end{align*}}
        \part{Tenemos lo siguiente:
          \begin{align*}
            \lambda a.(\lambda b.a \lambda b (\lambda a.a b)) &\equiv_{\alpha} \lambda a.(\lambda b.a \lambda b (\lambda a.ab)) [b:=c] \\
                                      &\equiv_{\alpha} \lambda a.(\lambda b.a \lambda c (\lambda a.ac)) [a:=d] \\
                                      &\equiv_{\alpha} \lambda a.(\lambda b.a \lambda c (\lambda d.dc))
          \end{align*}}
    \end{parts}
    \end{solution}
\end{questions}
\end{document}
