\documentclass[answers]{exam}

\usepackage[spanish]{babel}
\usepackage{minted}

\title{
  \huge Lenguajes de Programación \\[0.5cm]
  \LARGE Tarea 1
}

\author{
  Camila Alexandra Cruz Miranda \\
  Oscar Hernández Salinas \\
  José Ethan Ortega González \\
  David Pérez Jacome \\
}

\renewcommand{\solutiontitle}{\noindent\textbf{Solución:}\par\noindent}
\runningheadrule{}
\runningheader{Lenguajes de Programación}{Tarea 1}{\date{\today}}
\footer{}{Página \thepage\ de \numpages}{}

\begin{document}
\maketitle{}
\begin{questions}
  \question{Por cada uno de los miembros del equipo, buscar dos lenguajes de
    programación distintos e indicar los siguientes datos:}
  \begin{parts}
    \part{Nombre del lenguaje.}
    \part{Nombre de las o los diseñadores.}
    \part{Año de creación.}
    \part{Clasificación del lenguaje de acuerdo al: (1) Nivel, (2) Propósito,
      (3) Estilo/Paradigma.}
    \part{\}part{Ejemplo de sintaxis y semántica sencillo.}}
    \part{Ejemplo de biblioteca.}
    \part{Ejemplo de convención de programación (idiom).}
  \end{parts}
  \begin{solution}
    Aqui van los lenguajes de cada uno.
  \end{solution}

  \question{Define una función en Racket que calcule la longitud de un número
    natural, es decir, que calcule el número de dígitos que lo conforman.}
  \begin{solution}
    El código es el siguiente:
    \begin{minted}[frame=lines]{newlisp}
    (define (digitos n)
      (let ([m (floor (/ (log n) (log 10)))])
        (+ m 1)))
    \end{minted}
  \end{solution}

  \question{Define una función en Racket que filtre los números positivos de una
    lista. Es decir, dada una lista con números enteros, devolver una nueva
    lista con sólo números positivos.}
  \begin{solution}
      El código es el siguiente:
      \begin{minted}[frame=lines]{newlisp}
      (define (positivos l)
        (cond
          [(empty? l) l]
          [else (if (< (first l) 0)
                    (positivos (rest l))
                    (cons (first l) (positivos (rest l))))]))
      \end{minted}
    \end{solution}
\end{questions}
\end{document}
