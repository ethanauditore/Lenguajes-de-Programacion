\documentclass[12pt,letterpaper]{article}

\usepackage[spanish]{babel}
\usepackage{minted}
\usepackage{fullpage}
\usepackage[top=2cm, bottom=4.5cm, left=2.5cm, right=2.5cm]{geometry}
\usepackage{amsmath,amsthm,amsfonts,amssymb,amscd}
\usepackage{lastpage}
\usepackage{enumerate}
\usepackage{fancyhdr}
\usepackage{mathrsfs}
\usepackage{xcolor}
\usepackage{graphicx}
\usepackage{listings}
\usepackage{enumitem}
\usepackage{hyperref}

\hypersetup{%
  colorlinks=true,
  linkcolor=blue,
  linkbordercolor={0 0 1}
}

\setlength{\parindent}{0.0in}
\setlength{\parskip}{0.05in}

% Edit these as appropriate
\newcommand\curso{Lenguajes de Programación}
\newcommand\numerotarea{1}                  % <-- homework number

\pagestyle{fancyplain}
\headheight 53pt
\lhead{\small Camila Alexandra Cruz Miranda \\Oscar Hernández Salinas \\José Ethan Ortega González \\David Pérez Jacome}
\chead{\textbf{\curso \\Tarea \numerotarea}}
\rhead{Fecha de entrega: \\ \today}
\lfoot{}
\cfoot{}
\rfoot{\small\thepage}

\begin{document}

\begin{enumerate}
    \item Por cada uno de los miembros del equipo, buscar dos lenguajes de
    programación distintos e indicar los siguientes datos:
    \begin{enumerate}[label=(\alph*)]
        \item Nombre del lenguaje.
        \item Nombre de las o los diseñadores.
        \item Año de creación.
        \item Clasificación del lenguaje de acuerdo al: 
        \\(1) Nivel, \\(2) Propósito, \\(3) Estilo/Paradigma.
        \item Ejemplo de sintaxis y semántica sencillo.
        \item Ejemplo de biblioteca.
        \item Ejemplo de convención de programación (idiom).
    \end{enumerate}

    \item Define una función en Racket que calcule la longitud de un número natural, es decir, que calcule el número de dígitos que lo conforman.

    El código es el siguiente:
    \begin{minted}[frame=lines]{newlisp}
    (define (digitos n)
      (let ([m (floor (/ (log n) (log 10)))])
        (+ m 1)))
    \end{minted}

\item Define una función en Racket que filtre los números positivos de una
    lista. Es decir, dada una lista con números enteros, devolver una nueva
    lista con sólo números positivos.
    
      El código es el siguiente:
      \begin{minted}[frame=lines]{newlisp}
      (define (positivos l)
        (cond
          [(empty? l) l]
          [else (if (< (first l) 0)
                    (positivos (rest l))
                    (cons (first l) (positivos (rest l))))]))
      \end{minted}
      
\end{enumerate}
\end{document}
