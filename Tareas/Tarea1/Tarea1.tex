\documentclass[11pt,letterpaper]{article}

\usepackage[spanish]{babel}
\usepackage{minted}
\usepackage{fullpage}
\usepackage[top=2cm, bottom=4.5cm, left=2.5cm, right=2.5cm]{geometry}
\usepackage{amsmath,amsthm,amsfonts,amssymb,amscd}
\usepackage{lastpage}
\usepackage{enumerate}
\usepackage{fancyhdr}
\usepackage{mathrsfs}
\usepackage{xcolor}
\usepackage{graphicx}
\usepackage{listings}
\usepackage{enumitem}
\usepackage{hyperref}

\hypersetup{%
  colorlinks=true,
  linkcolor=blue,
  linkbordercolor={0 0 1}
}

\setlength{\parindent}{0.0in}
\setlength{\parskip}{0.05in}

% Edit these as appropriate
\newcommand\curso{Lenguajes de Programación}
\newcommand\numerotarea{1}                  % <-- homework number

\pagestyle{fancyplain}
\headheight 53pt
\lhead{\small Camila Alexandra Cruz Miranda \\Oscar Hernández Salinas \\José Ethan Ortega González \\David Pérez Jacome}
\chead{\textbf{\curso\\Tarea \numerotarea}}
\rhead{Fecha de entrega: \\ \today}
\lfoot{}
\cfoot{}
\rfoot{\small\thepage}
\headsep 1cm

\begin{document}


\begin{enumerate}
    \item Por cada uno de los miembros del equipo, buscar dos lenguajes de
    programación distintos e indicar los siguientes datos:
    \begin{enumerate}[label = (\alph*)]
        \item Nombre del lenguaje.
        \item Nombre de las o los diseñadores.
        \item Año de creación.
        \item Clasificación del lenguaje de acuerdo al: 
        \\(1) Nivel, \\(2) Propósito, \\(3) Estilo/Paradigma.
        \item Ejemplo de sintaxis y semántica sencillo.
        \item Ejemplo de biblioteca.
        \item Ejemplo de convención de programación (idiom).
    \end{enumerate}
        \begin{itemize}
          \item \textbf{Camila:}
                \begin{itemize}
                  \item \textbf{Nombre del lenguaje:} Wolfram Language.
                  \item \textbf{Nombre del diseñador:} Stephen Wolfram.
                  \item \textbf{Año de creación:} 1988, hace 33 años.
                  \item \textbf{Clasificación del lenguaje de acuerdo al:}
                        \begin{enumerate}
                          \item \textbf{Nivel:} Lenguaje de alto nivel.
                          \item \textbf{Propósito:} Propósito general.
                          \item \textbf{Estilo/Paradigma:} Multiparadigma,
                                símbolico, funcional y lógico.
                        \end{enumerate}
                  \item \textbf{Ejemplo de sintaxis y semántica sencillo:}
\begin{minted}[frame=lines]{mathematica}
(* Esto es un comentario. *)

4 + 3
(* = 7 *)

1 + 2 * (3 + 4)
(* = 15 *)
(* La multiplicación se puede omitir: 1 + 2 (3 + 4) *)

(* Las divisiones regresan números racionales: *)
3 / 2
(* = 3/2 *)
\end{minted}
                  \item \textbf{Ejemplo de biblioteca:}
                  \mint{mathematica}| fun = LibraryFunctionLoad["demo", "demo_I_I", {Integer}, Integer] |
                  \item \textbf{Ejemplo de convención de programación (idiom):}
\begin{minted}[frame=lines]{mathematica}
(*En vez de cambiar tu nube con la URL completa se puede hacer
con una palabra que la explique.*)
$CloudBase = "https://www.test.wolframcloud.com/"
Output: https://www.test.wolframcloud.com/

$CloudBase = "prd"
Output: https://www.test.wolframcloud.com/
\end{minted}
                \end{itemize}
                \begin{itemize}
                  \item \textbf{Nombre del lenguaje:} Python.
                  \item \textbf{Nombre del diseñador:} Guido van Rossum.
                  \item \textbf{Año de creación:} 1991, hace 33 años.
                  \item \textbf{Clasificación del lenguaje de acuerdo al:}
                        \begin{enumerate}
                          \item \textbf{Nivel:} Lenguaje de alto nivel.
                          \item \textbf{Propósito:} Propósito general.
                          \item \textbf{Estilo/Paradigma:} Multiparadigma,
                                orientado a objetos, funcional.
                        \end{enumerate}
                  \item \textbf{Ejemplo de sintaxis y semántica sencillo:}
\begin{minted}[frame=lines]{python}
# Este es un comentario
midpoint = 5

# Se crean dos listas vacías
lower = []; upper = []

# La identación importa para el control de los bloques de código
# separa los números en inferiores y superiores
for i in range(10):
    if (i < midpoint):
        lower.append(i)
    else:
        upper.append(i)

print("lower:", lower)
print("upper:", upper)
\end{minted}
                  \item \textbf{Ejemplo de biblioteca:}
                  \mint{python}|import matplotlib.pyplot as plt|

                  \item \textbf{Ejemplo de convención de programación (idiom):}
\begin{minted}[frame=lines]{python}
#Se puede cambiar el valor de dos variables por tuplas
x = True
y = False
x, y = y, x
x
# False
y
# True
\end{minted}
                \end{itemize}

          \item \textbf{Oscar}
                \begin{itemize}
                  \item \textbf{Nombre del lenguaje:} Swift.
                  \item \textbf{Nombre de los diseñadores:} Chris Lattner, Doug
                        Gregor, John McCall, Ted Kremenek, Joe Groff, y Apple
                        Inc..
                  \item \textbf{Año de creación:} 2010.
                  \item \textbf{Clasificación del lenguaje de acuerdo al:}
                        \begin{enumerate}
                          \item \textbf{Nivel:} Lenguaje de alto nivel.
                          \item \textbf{Propósito:} Programación de sistemas,
                                aplicaciones para móviles y de escritorio,
                                llegando a servicios en la nube.
                          \item \textbf{Estilo/Paradigma:} Multiparadigma
                                (Orientado a protocolos, objetos, funcional,
                                programación imperativa).
                        \end{enumerate}
                  \item \textbf{Ejemplo de sintaxis y semántica sencillo:}
\begin{minted}[frame=lines]{newlisp}
func printMessage(message: String) {
    println(message)
}
\end{minted}
                  \item \textbf{Ejemplo de biblioteca:}
                        \mint{swift}|import Foundation|
                  \item \textbf{Ejemplo de convención de programación (idiom):}
\begin{minted}[frame=lines]{newlisp}
for i in 1...5 {
  # Código
}
\end{minted}
                \end{itemize}
                \begin{itemize}
                  \item \textbf{Nombre del lenguaje:} R.
                  \item \textbf{Nombre de los diseñadores:} Ross Ihaka y Robert
                        Gentleman.
                  \item \textbf{Año de creación:} 1993.
                  \item \textbf{Clasificación del lenguaje de acuerdo al:}
                        \begin{enumerate}
                          \item \textbf{Nivel:} Lenguaje de alto nivel.
                          \item \textbf{Propósito:} Investigación científica,
                                machine learning, minería de datos,
                                investigación biomédica, bioinformática y
                                matemáticas financieras.
                          \item \textbf{Estilo/Paradigma:} Orientado a objetos.
                        \end{enumerate}
                  \item \textbf{Ejemplo de sintaxis y semántica sencillo:}
\begin{minted}[frame=lines]{newlisp}
while(condicion) {
  # Código
}
\end{minted}
                  \item \textbf{Ejemplo de biblioteca:}
                        \mint{r}|library(installr)|
                  \item \textbf{Ejemplo de convención de programación (idiom):}
\begin{minted}[frame=lines]{newlisp}
for (i in lista) {
  # Código
}
\end{minted}
                \end{itemize}
          \item \textbf{Ethan}
                \begin{itemize}
                  \item \textbf{Nombre del lenguaje:} Rust.
                  \item \textbf{Nombre de los diseñadores:} Graydon Hoare, Dave
                        Herman, Brendan Eich.
                  \item \textbf{Año de creación:} 2010.
                  \item \textbf{Clasificación del lenguaje de acuerdo al:}
                        \begin{enumerate}
                          \item \textbf{Nivel:} Lenguaje de medio nivel.
                          \item \textbf{Propósito:} Propósito general.
                          \item \textbf{Estilo/Paradigma:} Multiparadigma.
                        \end{enumerate}
                  \item \textbf{Ejemplo de sintaxis y semántica sencillo:}
                        \begin{enumerate}
                          \item \textbf{Sintaxis:} Comúnmente, para representar
                                una variable de alguno tipo numérico en varios
                                lenguajes se usa la palabra \texttt{int} o
                                \texttt{float}. En Rust, primero nos debemos de
                                preguntar de cuantos bits queremos nuestro
                                número, o si lo queremos con signo. A
                                continuación se muestra una tabla de todos los
                                tipos de números en Rust:

                                \begin{center}
                                  \begin{tabular}{lll}
                                    Tamaño & Con signo & Sin signo\\
                                    \hline
                                    8 bits & i8 & u8\\
                                    16 bits & i16 & u16\\
                                    32 bits & i32 & u32\\
                                    64 bits & i64 & u64\\
                                    128 bits & i128 & u128\\
                                    arch & isize & usize\\
                                  \end{tabular}
                                \end{center}

                                Entonces, para declarar debemos de usar la
                                palabra clave \texttt{let}, seguida de un
                                identificador, el simbolo :, el tipo de número,
                                el simbolo = y un valor.

                          \item \textbf{Semántica:} Siguiendo la sintaxis
                                anterior, obtenemos lo siguiente:
                                \mint{rust}|let n: i32 = 42;|
                                El significado dado es que a la variable
                                \texttt{n} se le asigna un tipo entero de 32
                                bits con signo, el cual tiene un valor numérico
                                de 42.
                        \end{enumerate}

                  \item \textbf{Ejemplo de biblioteca:} El equivalente a Rust a
                        una biblioteca sería un
                        \href{https://doc.rust-lang.org/book/ch07-01-packages-and-crates.html}{crate}.
                        Un crate bastante utilizado es
                        \href{https://crates.io/crates/rand}{rand}, que nos
                        permite generar números aleatorios.

                  \item \textbf{Ejemplo de convención de programación (idiom):}
                        Rust no tiene constructores como tal, en general, se usa
                        una función estática para crear un nuevo objeto. Por
                        ejemplo:
\begin{minted}[frame=lines]{rust}
struct Rectangulo {
    base: u32,
    altura: u32,
}

impl Rectangulo {
    fn new() -> Rectangulo {
        Rectangulo {
            base: 0u32,
            altura: 0u32,
        }
    }
}
\end{minted}
                \end{itemize}
                \begin{itemize}
                  \item \textbf{Nombre del lenguaje:} C++.
                  \item \textbf{Nombre del diseñador:} Bjarne Stroustrup.
                  \item \textbf{Año de creación:} 1983.
                  \item \textbf{Clasificación del lenguaje de acuerdo al:}
                        \begin{enumerate}
                          \item \textbf{Nivel:} Lenguaje de medio nivel.
                          \item \textbf{Propósito:} Propósito general.
                          \item \textbf{Estilo/Paradigma:} Multiparadigma.
                        \end{enumerate}
                  \item \textbf{Ejemplo de sintaxis y semántica sencillo:}
                        \begin{enumerate}
                          \item \textbf{Sintaxis:} Para poder imprimir algún
                                texto a la consola, utilizamos
                                \texttt{std::cout} seguido del operador
                                \texttt{\<\<} y los caracteres o números como
                                salida. Cabe mencionar que, para utilizar
                                \texttt{std::cout}, es necesario incluir la
                                directiva del preprocesador para poder utilizar
                                la biblioteca \texttt{iostream}
                          \item \textbf{Semántica:} Siguiendo la sintaxis
                                anterior, obtenemos lo siguiente:
\begin{minted}[frame=lines]{cpp}
#include <iostream>

int main()
{
  std::cout << "Hola, Mundo!";
    return 0;
}
\end{minted}
                        \end{enumerate}
                  \item \textbf{Ejemplo de biblioteca:} Una de las bibliotecas
                        más usadas es \texttt{iostream}, que define a los
                        objetos de entrada y salida estándar.
                  \item \textbf{Ejemplo de convención de programación (idiom):}
                        Al igual que en C, en C++ las llaves de una función se
                        escriben en una línea aparte:
\begin{minted}[frame=lines]{cpp}
#include <iostream>

int main()
{
  std::cout << "Hola, Mundo!";
    return 0;
}
\end{minted}

                \end{itemize}
          \item \textbf{David}
                \begin{itemize}
                  \item \textbf{Nombre del lenguaje:} Java.
                  \item \textbf{Nombre del diseñador:} Sun Microsystems.
                  \item \textbf{Año de creación:} 1995.
                  \item \textbf{Clasificación del lenguaje de acuerdo al:}
                        \begin{enumerate}
                          \item \textbf{Nivel:} Lenguaje de alto nivel.
                          \item \textbf{Propósito:} Propósito general.
                          \item \textbf{Estilo/Paradigma:} Orientación a
                                objetos, crear aplicaciones (empresariales, web
                                y escritorio), permite ejecutar un mismo programa
                                en múltiples sistemas operativos y hace posible
                                ejecutar el código en sistemas remotos de manera
                                segura.
                        \end{enumerate}
                  \item \textbf{Ejemplo de sintaxis y semántica sencillo:}
                        \begin{enumerate}
                          \item \textbf{Sintaxis:} Un ejemplo de sintaxis es el
                                siguiente:
\begin{minted}[frame=lines]{java}
//Ejemplo en java
int var = 10;
\end{minted}
                          \item \textbf{Semántica:} En este caso usando este
                                mismo ejemplo tenemos una variable de nombre
                                \textbf{"var"} de tipo \textbf{int} que se
                                refiere a tipo entero con un valor de
                                \textbf{10}.
                        \end{enumerate}
                  \item \textbf{Ejemplo de biblioteca:} Java provee una amplia
                        funcionalidad para crear nuevas aplicaciones. Un ejemplo
                        es \textbf{javax.jms}.

                  \item \textbf{Ejemplo de convención de programación (idiom):}
                        Uno de los idioms que posee el lenguaje de programación
                        java es los nombres de las clases deben ser sustantivos,
                        en mayúsculas y minúsculas, con la primera letra de cada
                        palabra interna en mayúscula. El nombre de las
                        interfaces también debe estar en mayúscula (la primera)
                        al igual que los nombres de las clases. Use palabras
                        completas y debe evitar acrónimos y abreviaturas. Por
                        ejemplo:
\begin{minted}[frame=lines]{java}
public class Lenguaje {
    // código
}
\end{minted}

                \end{itemize}
                \begin{itemize}
                  \item \textbf{Nombre del lenguaje:} Ruby.
                  \item \textbf{Nombre del diseñador:} Yukihiro Matsumoto.
                  \item \textbf{Año de creación:} 1995.
                  \item \textbf{Clasificación del lenguaje de acuerdo al:}
                        \begin{enumerate}
                          \item \textbf{Nivel:} Lenguaje de alto nivel.
                          \item \textbf{Propósito:} Principalmente en la
                                programación web a través del framework Ruby on
                                Rails.
                          \item \textbf{Estilo/Paradigma:} Orientado a objetos.
                        \end{enumerate}
                  \item \textbf{Ejemplo de sintaxis y semántica sencillo:}
                        \begin{enumerate}
                          \item \textbf{Sintaxis:} Un ejemplo de sintaxis es el
                                siguiente:
\begin{minted}[frame=lines]{ruby}
#Ejemplo en ruby
x = 10;
\end{minted}
                          \item \textbf{Semántica:} En este caso, debemos de
                                leer esta asignación como el definir una
                                variable de nombre \textbf{"x"} a la que le
                                estamos asignando el valor de \textbf{10}.
                        \end{enumerate}
                  \item \textbf{Ejemplo de biblioteca:} Las dependencias para
                        las aplicaciones de Ruby se declaran en un archivo
                        Gemfile. Por ejemplo:
\begin{minted}[frame=lines]{ruby}
source "https://rubygems.org"
gem "rails"
\end{minted}

                  \item \textbf{Ejemplo de convención de programación (idiom):}
                        En Ruby, para declarar constantes se utilizan tenemos
                        las siguientes opciones:
                        \mint{ruby}|THIS\_CONSTANT, This\_Constant, ThisConstant.|
                \end{itemize}
        \end{itemize}

    \item Define una función en Racket que calcule la longitud de un número natural, es decir, que calcule el número de dígitos que lo conforman.

    El código es el siguiente:
    \begin{minted}[frame=lines]{newlisp}
    (define (digitos n)
      (let ([m (floor (/ (log n) (log 10)))])
        (+ m 1)))
    \end{minted}

\item Define una función en Racket que filtre los números positivos de una
    lista. Es decir, dada una lista con números enteros, devolver una nueva
    lista con sólo números positivos.
    
      El código es el siguiente:
      \begin{minted}[frame=lines]{newlisp}
      (define (positivos l)
        (cond
          [(empty? l) l]
          [else (if (< (first l) 0)
                    (positivos (rest l))
                    (cons (first l) (positivos (rest l))))]))
      \end{minted}
      
\end{enumerate}
\end{document}
